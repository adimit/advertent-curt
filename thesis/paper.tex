\documentclass{scrartcl}

\usepackage[english]{babel}
\usepackage[utf8x]{inputenc}
\usepackage{bbm,qtree,natbib,stmaryrd,aleks,linguex}

\newcommand{\abbr}{\textsf}
\newcommand{\pn}{\textsc}
\newcommand{\url}[1]{\texttt{http://#1}}
%names
\newcommand{\curt}{\pn{Curt}}
\newcommand{\prol}{\pn{Prolog}}


\bibliographystyle{plainnat}

\author{Aleksandar Dimitrov}
\title{From Questions to Queries -- Enhancing the Curt System by a Theory and Implementation of Embedded WH-Questions}
\begin{document}

\maketitle
\tableofcontents

\section{Choosing Grounds} % FIXME title sucks

\subsection{The Curt System}

\subsection{General Comparison of the Curt System}

This section will briefly discuss how the features of the Curt system compare to
other similar systems or systems with the same goal, namely implementing a
system that is able to model a logic representation of human language.

\subsection{Early Attempts: \pn{Shrdlu} and \pn{Cyc}}
%TODO: need some reference for this section!

Interest in deep semantic parsing of natural language has existed since the very
beginning of the informational age. One of the first systems to achieve public
recognition, the \pn{Shrdlu} system 


\section{Concepts}

After having elaborated on the grounds of the implementation, we will now
display the implementation's underlying concepts. % FIXME this sentence is too short

%OUTLINE Presenting a conceptual overview, while arguing for and against stuff.

First we will present a brief overview of the considerations regarding the
choice of an appropriate formalism for the system in section \ref{sec:formal}
to then compare it with already existing implementations of question answering
systems in \ref{sec:othercrap}.

\subsection{Formalism}\label{sec:formal}

Most of the traditional literature in natural language semantics follows
\cite{ptq} and thus also uses its underlying formalism, \emph{Intensional Logic}
(henceforth \abbr{IL}). \cite{gs:sqpa} (and subsequent publications) however use
two-sorted type logic (henceforth \abbr{Ty2}) as proposed in
\cite{gallin:ty2}. \cite{z:ilty2} discusses the differences and similarities in expressive power
between \abbr{Ty2} and \abbr{IL} and comes to the conclusion that the set of
expressions expressible in the former but not in the latter might not be
relevant for the treatment of natural language semantics. Although he proposes
a translation scheme between both languages, translation from \abbr{Ty2} to
\abbr{IL} seems to sometimes produce very complex \abbr{IL} terms.

Since this paper is not concerned with possible extensions of the underlying
theories of treating natural language questions, but rather with their
implementation in the programming language \pn{Prolog} several additional
constraints have to be considered. Among them the \emph{feasibility} of any
possible approach takes highest precedence.

While the \pn{Curt} system is unique in its approach to deep semantic analysis, its
architecture has been crafted in a pragmatic manner. The authors chose to use
external tools where it was possible in order to avoid reimplementation of
existing functionality and be able to focus on novel parts of the
implementation\footnote{The name of the \pn{Curt} system reflects this choice.
It stands for \emph{\textbf{C}lever \textbf{u}se of \textbf{r}easoning
\textbf{t}ools}.}. While there exist inferencing tools for modal
logic\footnote{See %todo todo and todo 
for discussions. The interested reader is also referred to \pn{Molle}
(\url{molle.sf.net}), a modal theorem prover that comes with a graphical user
interface that allows interactive investigation of the constructed indices.},
a restructuring of the \pn{Curt} system would to take advantage of them would be
beyond the scope of this thesis. Another design decision of the \pn{Curt} system
was to use \prol as an implementation language. While \prol is a general
purpose logic programming language, it is rooted in first-order predicate logic
and does not explicitly support modal logic\footnote{But see%todo, todo and todo
for a discussion of the implementation of modal logic into \prol}.

It was therefore decided to avoid modal logic and implement a pragmatic approach
to representing the indices of higher-order logic in \prol in section
\ref{sec:indices}.

\cite{g:inqs} presents an approach to implementing inquisitive semantics in
predicate logic.

\subsection{Existing Implementations of Similar Systems}\label{sec:othercrap}

Godot, all that statistical crap and Pereira's CHAT-80.

\section{Implementation}

\subsection{An Ontology of Embedding Verbs}

Karttunen's List.

\subsubsection{Extensional WH-embedding Verbs}

Like \emph{know}.

\subsubsection{Intensional WH-embedding Verbs}

Like \emph{believe}.

\subsection{Modality and Possible Worlds}
\label{sec:indices}
\subsection{Treatment of Baker-Ambiguities}

\section{Discussion}

\bibliography{aleksbib}


\end{document}

