\documentclass[a4paper,fontsize=12pt]{article}

\usepackage[english]{babel}
\usepackage[utf8x]{inputenc}
\usepackage{amsthm,bbm,qtree,natbib,stmaryrd,amsmath,amssymb,aleks,linguex}

\newcommand{\abbr}{\textsf} % how to render abbreviations like IL or Ty2
\newcommand{\stress}{\textbf} % put stress somewhere (as opposed to \emph)
\newcommand{\term}{\textsf} % introduce a term
\newcommand{\code}{\texttt} % layout of in-line code snippets
%\newcommand{\quote}[1]{\textit{``#1''}} % a direct quote from a paper
\newcommand{\pn}{\textsc} % how to render proper names like Shrdlu
\newcommand{\url}[1]{\code{http://#1}} % how urls appear

% commonly used phrases
\newcommand{\Disc}{\ensuremath{\mathcal{D}}} % the discourse
\newcommand{\wh}{\textsc{wh}} %

%names
\newcommand{\curt}{\pn{Curt}\mbox{ }}
\newcommand{\acurt}{\pn{Advertent Curt}\mbox{ }}
\newcommand{\prol}{\pn{Prolog}\mbox{ }}

%theorem environments
\theoremstyle{remark} \newtheorem*{termin}{Definition} % terminology. Could use a
%better title fixme

\bibliographystyle{plainnat}

\author{Aleksandar Lyubenov Dimitrov\\
$\langle$\href{mailto:aleks.dimitrov@googlemail.com}{aleks.dimitrov@googlemail.com}$\rangle$}
\title{From Questions to Queries -- Enhancing the Curt System by a Theory and Implementation of Embedded WH-Questions}

\usepackage[pdftex,colorlinks,citecolor=magenta]{hyperref}
\begin{document}
\maketitle
\tableofcontents
\newpage

\section*{Introduction}

%fixme: to be formalized
Where we use ``natural language'', usually the English language will be treated
in practice, though it would be nice to come up with treatments of other
languages. We abstract away from pragmatic affairs and concern ourselves with
mostly the purely semantic/logic domain, so this should be feasible to achieve.
\subsection*{Acknowledgements}

\section{Concepts}

% fixme: formalize
A brief overview of the premisses for the theory (which itself lays ground to
the implementation).

\subsection{The Curt System} \label{sec:curt}

\pn{Curt}, a system written in \prol that is able to interpret basic utterances
in natural language and store them in expressions of first order logic, is presented
in \cite{blackburnbos:cl1}. While relatively limited when compared to other
systems presented in \ref{sec:comparision} in
terms of lexical and grammatical coverage it is able to perform basic checks for
\emph{consistency} and \emph{informativity} of its input\footnotemark, within
the bounds of the tools it uses.

% FIXME: don't really like this footnote
\footnotetext{Here, the notion of \emph{consistency} is meant as \emph{validity}
of the input and \emph{informativity} denotes \emph{validity}. Thus, if
\Disc, the discourse so far and $\phi$ a new input token (fully parsed
sentence), then consistency means $\Disc \to \phi$ holds and informativity
means $\Disc \nvDash \phi$.}


It can thus detect incoherent discourse and would reject a sequence of
sentences such as \emph{``Mary likes every man. Peter is a man. Mary does not
like Peter''}, claiming it found a contradiction in its input. For every input
token, \pn{Curt} will also try to infer whether the current discourse would
entail the information given in its newest token and is thus irrelevant with
regard to previous utterances -- effectively allowing it to
reduce the size of its internal representation of the discourse by not adding
clauses to it which would not contribute to the overall informational value
already present in the program.

The system is introduced during the final chapter of \cite{blackburnbos:cl1},
incrementally increasing in features and complexity. Support for external
inferencing via theorem provers and model builders is added in a step-by-step
fashion and finally an ontology and a straight-forward way of parsing and
answering direct questions for noun phrases conclude its development. The theory
and implementation of this paper is meant as a replacement and enhancement for
the latter feature. This will also happen in an incremental manner and new
features will be introduced one at a time.

\subsection{General Comparison of the Curt System}\label{sec:comparision}

This section will briefly discuss how the features of the Curt system compare to
other similar systems or systems with the same goal, namely implementing a
system that is able to model a logic representation of human language.

\subsubsection{Early Attempts: \pn{Shrdlu} and \pn{Cyc}}
%TODO: need some reference for this section!

Interest in deep semantic parsing of natural language has existed since the very
beginning of the informational age. One of the first systems to achieve public
recognition, the \pn{Shrdlu} system 

While the \pn{Curt} system is unique in its approach to deep semantic analysis, its
architecture has been crafted in a pragmatic manner. The authors chose to use
external tools where it was possible in order to avoid reimplementation of
existing functionality and be able to focus on novel parts of the
implementation\footnote{The name of the \pn{Curt} system reflects this choice.
It stands for \emph{\textbf{C}lever \textbf{u}se of \textbf{r}easoning
\textbf{t}ools}.}. This also allows for great flexibility in adopting new
solutions and improvements made in the field of inferencing tools for first
order predicate logic.

\subsection{Extending the Curt System}

At the end of \cite{blackburnbos:cl1}, \curt is able to cope with a small
grammar and provides some features for interacting with humans in natural
language. It does only understand basic extensional transitive verbs - modal
verbs or the treatment of questions are not implemented. Anaphora resolution
still remains an open issue. The grammar is rather limited and \curt only comes
with a basic vocabulary oriented around the setting of a popular movie.

The aim of this paper is to introduce the reader into question semantics as it
will be relevant for implementing a basic capability of treating intensional
verbs like `to \stress{know} whether something is the case' and `to
\stress{believe} that something might be the case' and then present such an
implementation within the framework of the \curt system. Further and more
advanced investigations will be dealt with at the end of this discussion or left
for further research.

\section{Theory}

\subsection{Terminology}

Following \cite{gs:q}, we will not use the natural natural language word
``question'' but distinguish between \term{questions},
\term{interrogative sentences} and \term{interrogative acts} in the following
way:

\begin{termin}
  An \term{interrogative} is a certain type of sentence in natural language
  distinguished by certain features\footnote{To quote \cite{gs:q}: \quote{[An
  interrogative sentence is] characterized by word order, intonation, question
  mark, the occurrence of interrogative pronouns}.}
  whereas an \term{interrogative act} refers to the utterance of an
  \emph{interrogative}. Finally, a 
  \term{question} will denote the semantic entity uttered by the
  \emph{interrogative act}, whichever form it may have.
\end{termin}

\abbr{Ty2} is two-sorted type logic\footnote{See \cite{gallin:ty2} for a
discussion and \cite{gs:sqpa} for its usage in describing the semantics of
questions}. \abbr{IL} is intensional logic\footnote{As used by \cite{ptq}}. 
An embedded \wh-phrase looks
like:%fixme

\ex. John \stress{knows} \emph{whether} good examples are hard to come up with.

After having elaborated on the grounds of the implementation, we will now
display the implementation's underlying concepts. % FIXME this sentence is too short

%OUTLINE Presenting a conceptual overview, while arguing for and against stuff.

First we will present a brief overview of the considerations regarding the
choice of an appropriate formalism for the system in section \ref{sec:formal}
to then compare it with already existing implementations of question answering
systems in \ref{sec:othercrap}.

\subsection{Formalism}\label{sec:formal}

% FIXME if we have a `Terminology' section, all those `henceforths' aren't really
% necessary anymore.

Most of the traditional literature in natural language semantics follows
\cite{ptq} and thus also makes use of its underlying formalism,
\emph{Intensional Logic} (henceforth \abbr{IL}). \cite{gs:sqpa} (and many subsequent
publications) however also use two-sorted type logic (henceforth \abbr{Ty2}) as
proposed in \cite{gallin:ty2}. \cite{z:ilty2} discusses the differences and
similarities in expressive power between \abbr{Ty2} and \abbr{IL} and comes to
the conclusion that the set of expressions contained in the former but not in
the latter might not be relevant for the treatment of natural language
semantics. Although he proposes a translation scheme between both languages,
translation from \abbr{Ty2} to \abbr{IL} seems to sometimes produce very complex
\abbr{IL} terms and is therefore avoided here. In fact, \abbr{IL} will be only
marginally relevant for the presentation of \acurt -- only the type system
presented in \cite{ptq} is carried over.

% OK till now, but the rest is a fixme :-(

Since this paper is not concerned with possible extensions of the underlying
theories of treating natural language questions, but rather with their
implementation in the programming language \pn{Prolog} several additional
constraints have to be considered. Among them the \emph{feasibility} of any
possible approach takes highest precedence.

As was already mentioned in \ref{sec:curt}, Curt is based on an
implementation and the use of external tools only capable of treating first
order logic -- an extensional framework. \prol on the other hand, is a language
that can very well make use of higher order formalisms for
calculation\footnote{The two most popular predicates in \prol used for
implementing higher-order logic are
\code{call/n} and \code{apply/3}. This paper is based on \pn{SWI-Prolog} which
provides an implementation of the former, but not of the latter. See
\cite{naish:prolhio} for a discussion of higher order functional programming in
\prol.}.

While there exist inferencing tools for modal logic\footnote{See %todo todo and todo 
for discussions. The interested reader is also referred to \pn{Molle}
(\url{http://molle.sf.net}), a modal theorem prover that comes with a graphical user
interface that allows interactive investigation of the constructed indices.},
a restructuring of the \pn{Curt} system to take advantage of them would be
beyond the scope of this thesis.

It was therefore decided to avoid modal logic and implement a pragmatic approach
to representing the indices of higher-order logic in \prol in section
\ref{sec:indices}.

\cite{g:is} presents an approach to implementing inquisitive semantics in
predicate logic.

\subsection{On the Proper Treatment of Questions}

Question Semantics is a much-disputed but often overlooked field of modern
semantic research. Often question semantics has been reduced to an issue of
pragmatics, treating direct questions (interrogatives) as nothing more than a
paraphrase of indirect questions (embedded interrogatives).\cite{tichy}
even goes as far as claiming equality in semantic value between
indicatives and interrogatives. The benefit of such an approach would be the
strict reliance on traditional analyses, where interrogatives would only infer a
different `attitude' in the speaker's utterance. Putting this attitude into
context and making it yield the correct results would then be left to
pragmatics.
This approach has been challenged and proven wrong by \cite{gs:q},
who also argue that the proper treatment of questions would require extensions
to the existing logic frameworks used in natural language semantics semantics.
More recent publications, such as \cite{gal:tmbi}
address this topic and aim to introduce notions of \term{information change
potential} as opposed to the traditional focus of natural language semantics on
\term{truth values}.

Nevertheless, progress has been very lively since the first
formal approaches to the field started with \cite{hamblin:q} and other
publications such as % todo
. \cite{gs:q} provide a very descriptive overview of the field of question
semantics

\subsubsection{Classification of Interrogatives}

When treating questions one will be confronted with many different
categorizations of questions. One of the first distinctions to make is the one
between \term{embedded questions} as in \ref{ex:embed} and \term{direct
questions} shown in \ref{ex:direct}.

\ex. \a. John knows \emph{whether this sentence contains an embedded
question}.\label{ex:embed}
\b. Does this sentence pose a direct question?\label{ex:direct}

As indicated already by the title, this paper will concern itself with the
treatment of the former and only briefly comment on the latter. The reasons for
this are numerous  %fixme I need a synonym for that
and this section is going to name the most important ones.

Early literature, such as \cite{karttunen:1977} only regards embedded
\wh-phrases, treating direct questions as semantically equivalent. To quote
Karttunen:

\begin{quote}A direct question can be treated as semantically equivalent to a
certain kind of declarative sentence containing the corresponding indirect
question embedded under  a suitable `performative' verb.\end{quote}

Using this strategy, Karttunen escapes the treatment of direct  questions and
thus gives an approach for indirect questions only.

\subsection{An Ontology of Embedding Verbs}

\cite{karttunen:1977} introduces means of classification for \wh-embedding
verbs, the nature of which \cite{lahiri:diss} covers in a much broader scope.
The different verb classes ask for different strategies of treating the
\wh-clauses they embed and also pose different restrictions on the type of
clause they expect.

Our aim is to improve coverage of grammatical structures and thus when faced
with the necessity to make compromise about the complexity of our grammar, we
will choose overgeneration in favor of unmotivated complexity of the grammar
itself.

\subsubsection{Extensional WH-embedding Verbs}

Like \emph{know}.

We first derive know(john,all(X,imp(woman(X),love(X,john)))) from the syntax of
john knows whether every woman loves john, then we see if
all(X,imp(woman(X),love(X,john))) is valid (uninformative) and ridicule John, if
it's not valid and not consistent, assert his knowledge of the fact that it's
not true. If it's true, we assert he knows that.

For whether we can ridicule him if it's not true (inconsistent).
The MB/TP do a comparison \Disc to all(X,imp(woman(X),love(X,john)))

\subsubsection{Intensional WH-embedding Verbs}

Like \emph{believe}.

\section{Implementation}

\subsection{Existing Implementations of Similar Systems}\label{sec:othercrap}

Enhancing \curt by an implementation dealing with questions in natural language
sparks interest in whether there have been similar efforts made in the past. No
further extensions to the \curt system aiming at comparable goals are known to the
author at the time of writing. However, there is a lively interest in providing
so-called question answering or QA-systems since the early stages of computer
science. % fixme; cite some ancient 60's paper
This section discusses a few of the more elaborate approaches: first an early
but already quite impressive implementation is shown, followed by a newer approach
using embodied agents and finally an overview over the many database-based
QA-systems is given. Finally, all this is compared what the current extension to
\curt can do.


\subsubsection{Godot, the Talking Robot}

\subsubsection{CHAT-80}

\subsubsection{Modern Intensional Database Approaches}

\subsubsection{Advertent Curt}

\acurt lends its name from the fact that it is not only aware of its own
knowledge and presuppositions, but `knows' about what individuals in its
model of the world are aware of. It is not in this sense a question
\emph{answering} system, but it may be seen as an early effort towards providing
such a system on a more sophisticated scale. While section \ref{sec:altq} shows
how direct questions can be integrated into the system with relative ease, focus
is still on providing a feasible method of implementing intensionality in what
is mostly a first-order logic computing environment.

It is also not constrained to being useful in a particular domain, like
\pn{CHAT-80} is.

\subsection{Modality and Possible Worlds}
\label{sec:indices}

We have seen that some embedding verbs may not be interpreted without
implementing a higher order formalism that is able to model the notion of
\emph{possible} and \emph{accessible} worlds. While foobar %fixme! Nelken &
% Francez 2001
argue for an extensional treatment of questions, they already % n & f 2006
retract their claims and admit that a ``minimal account of intensionality'' has
to be assumed. While they propose a two-world system later adopted in
\cite{g:is}, this is not going to suffice if we want to model each individual's
knowledge in order to be able to discriminate people's knowledge and beliefs.
%fixme bla bla

\subsubsection{Enhancing Storage}

\curt parses sentences of natural language using an enhanced version of \pn{Cooper
storage}\footnote{See \cite{cooper:storage2} for the original implementation.} as
presented in \cite{keller:storage}. \cite{blackburnbos:cl1} call this version
\pn{Keller storage}. That name will be used from here % FIXME sucks. Please change
to refer to this particular implementation as presented in
\cite{blackburnbos:cl1}. Although \pn{Keller storage} is already quite capable
of dealing with more complex constituents such as nested noun phrases, it needs
some enhancements in order to be able to treat verbs embedding whole
sentences or \wh-phrases correctly. The aim is to enable \curt's implementation of
\pn{Keller storage} to deal with examples such as \ref{ex:kel1}\footnotemark.

\ex. \label{ex:kel1} Mia knows whether Vincent snorts.\\
\Tree [.t [.T Mia ] [.IV [.{IV/$\bar{t}$} knows ] [.{$\bar{t}$} whether [.t [.T
Vincent ] [.IV snorts ] ] ] ] ]

\footnotetext{Note that \ref{ex:kel1} is shown using the same syntactic type system used in
\cite{gs:sawhq}, which extends the system given in \cite{ptq} by $\bar{t}$ to
denote embedded \wh-constituents.}

\subsection{Treatment of Baker Ambiguities}

\subsection{Alternative Questions -- Inquisitive Semantics}\label{sec:altq}

\cite{g:is} introduces a novel approach to treating questions in natural
semantics: \pn{Inquisitive Semantics}. He presents a new logic
formalism aimed at providing a wider coverage of the intentions of question in
human language. The main source of the ``inquisitiveness'' of statements is
given by the interpretation of \emph{disjunction}. We will now show how to
incorporate this novel approach into the \pn{Curt} system.

We will, however depart from the path we have been following throughout this
paper and first implement a theory for direct questions. Only then will we extend
it to also cover embedded alternative questions.

\section{Discussion}

We have seen, the Curt system is quite capable of treating questions this way.
Having implemented modality and possible worlds in such a fashion, let's go on
and implement some even cooler stuff (tense, aspect, negation, comparatives,
anaphora). Basing all this on a nice implementation of DRT (discussed in
\cite{kampreyle:drt}) like the one given in \cite{blackburnbos:cl2} might be
interesting.

\section{Appendix}

\subsection{The Source Code}

\subsection{Whatnot}

\bibliography{aleksbib}

\end{document}

