\documentclass{article}

\usepackage[english]{babel}
\usepackage{qtree}
\usepackage[utf8x]{inputenc}
\usepackage{linguex}
\usepackage{natbib}

\bibliographystyle{plainnat}

\author{Aleksandar Dimitrov}
\title{Questioning Discourse in Prolog: From Questions to Queries}

\begin{document}

\maketitle

Discourse Representation Theory (DRT), as presented in \citet{kampreyle:drt}
provides a powerful and flexible approach to capturing the meaning of natural
language and storing information in tidy data structures.
\citep{blackburnbos:cl2} further provide us with an implementation of DRT in
Prolog. Although unfinished, their results may already prove useful to an
attempt of an implementation of a generic way to retrieve information from
Discourse Representation Structures (DRS) in form of either absolute truth
values (answers to alternative questions) in questions like \ref{qex:yesno} or a
possible set of individuals as in \ref{qex:set}.

\ex. 
\a. Is Mary pregnant? \label{qex:yesno}
\b. Who is coming to the party? \label{qex:set}

The problem of treating questions has generally been recognized as being
non-trivial and there exist several attempts to explain the semantics of
questions in natural language, the classical examples being \citet{hamblin:1973}
with a first rather simple attempt, \citet{karttunen:1973} and \citet{gs:1984a},
who both take a much more sophisticated approach to the proper treatment of
questions.

One of the main challenges of this work will be the unification of classical
linguistic theories with the newer DRT. Both originate from entirely different
mindsets. While the classic theory according to \citep{karttunen:1973} is based
on Montague's analysis of English (in \citet{ptq}) and thus presupposes
intensional logic, DRT introduces its own formalism and does so far not treat
intensionality. Trying to let the classic theory operate on DRT in a naïve way
will thus fail. This makes it imperative to find a feasible approach to let
question operators work with DRSs in a well-defined way.

As a further challenge, implementing presuppositions and/or modality might prove
to be an interesting task. 

\ex. \label{presups}
\a. Is Jon pregnant? \label{presup}
\b. Can Jon be pregnant? \label{modal}

In \ref{presups} we see two sentences which should be answered on grounds of
presuppositions - Jon is presumably male and can thus not be pregnant. In
\ref{presup}, an answer to the question may not even need to be proved against
the current model, while a sentences as \emph{Jon became pregnant} should be
rejected by the system. Cases like \ref{modal} might be much harder to properly
treat, since they involve modal verbs.

While treating presuppositions in DRT is not new\footnote{\citet[Chap.
4]{blackburnbos:cl2} provide a discussion on the implementation in Prolog},
modality has so far not received great attention\footnote{Albeit a few papers do
seem to exist. See \citet{frank96context}.}. Since questions in natural language
rely quite a lot on presuppositions, implementing at least a primitive method
seems to be an enticing opportunity to further explore the treatment of natural
language questions to DRSs.

Based on \citet{jaeger:1997} and possibly others, one might also want to
consider implementing \emph{informative questions}, that is, questions that not
only query for informations, but also \emph{provide} information to the system.

\bibliography{aleksbib}

\end{document}

